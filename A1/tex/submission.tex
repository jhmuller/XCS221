% This contents of this file will be inserted into the _Solutions version of the
% output tex document.  Here's an example:

% If assignment with subquestion (1.a) requires a written response, you will
% find the following flag within this document: <SCPD_SUBMISSION_TAG>_1a
% In this example, you would insert the LaTeX for your solution to (1.a) between
% the <SCPD_SUBMISSION_TAG>_1a flags.  If you also constrain your answer between the
% START_CODE_HERE and END_CODE_HERE flags, your LaTeX will be styled as a
% solution within the final document.

% Please do not use the '<SCPD_SUBMISSION_TAG>' character anywhere within your code.  As expected,
% that will confuse the regular expressions we use to identify your solution.

\def\assignmentnum{1 }
\def\assignmentname{Sentiment Analysis}
\def\assignmenttitle{XCS221 Assignment \assignmentnum --- \assignmentname}
\input{macros}
\begin{document}
\pagestyle{myheadings} \markboth{}{\assignmenttitle}

% <SCPD_SUBMISSION_TAG>_entire_submission

This handout includes space for every question that requires a written response.
Please feel free to use it to handwrite your solutions (legibly, please).  If
you choose to typeset your solutions, the |README.md| for this assignment includes
instructions to regenerate this handout with your typeset \LaTeX{} solutions.
\ruleskip

\LARGE
1.d
\normalsize

% <SCPD_SUBMISSION_TAG>_1d
\begin{answer}
  % ### START CODE HERE ###
  Example 1:
=== a perfectly competent and often imaginative film that lacks what little lilo \& stitch had in spades -- charisma .
Truth: 1, Prediction: -1 [WRONG]

Note the word "lack" was in the text with very negative weights, but in this case I think the "lacks" refers to a different movie


Example 2:
=== 'it's painful to watch witherspoon's talents wasting away inside unnecessary films like legally blonde and sweet home abomination , i mean , alabama . '
Truth: -1, Prediction: 1 [WRONG]


Here the words "sweet" and "home" provided very positive weights but they were just the names of another movie


Example 3:
=== wickedly funny , visually engrossing , never boring , this movie challenges us to think about the ways we consume pop culture .
Truth: 1, Prediction: -1 [WRONG]

Here "never" and "boring" gave strong negative values, but note they were used together to mean a positive sentiment.


Example 4:
=== even a hardened voyeur would require the patience of job to get through this interminable , shapeless documentary about the swinging subculture .
Truth: -1, Prediction: 1 [WRONG]

Both "patience" and "shapeless" had positive weights.  Maybe not so strange for "patience" but wrong for how it is used.
Hard to see why "shapeless" has a positive weight

Example 5:
=== wedding feels a bit anachronistic . still , not every low-budget movie must be quirky or bleak , and a happy ending is no cinematic sin .
Truth: 1, Prediction: -1 [WRONG]

Odd, but two of the most negative weights here were "feel" and "ending"

  
  % ### END CODE HERE ###
\end{answer}
% <SCPD_SUBMISSION_TAG>_1d
\clearpage

\LARGE
1.f
\normalsize

% <SCPD_SUBMISSION_TAG>_1f
\begin{answer}
  % ### START CODE HERE ###
  % ### END CODE HERE ###
\end{answer}
% <SCPD_SUBMISSION_TAG>_1f
\clearpage

% <SCPD_SUBMISSION_TAG>_entire_submission

\end{document}
