% This contents of this file will be inserted into the _Solutions version of the
% output tex document.  Here's an example:

% If assignment with subquestion (1.a) requires a written response, you will
% find the following flag within this document: <SCPD_SUBMISSION_TAG>_1a
% In this example, you would insert the LaTeX for your solution to (1.a) between
% the <SCPD_SUBMISSION_TAG>_1a flags.  If you also constrain your answer between the
% START_CODE_HERE and END_CODE_HERE flags, your LaTeX will be styled as a
% solution within the final document.

% Please do not use the '<SCPD_SUBMISSION_TAG>' character anywhere within your code.  As expected,
% that will confuse the regular expressions we use to identify your solution.

\def\assignmentnum{5 }
\def\assignmentname{Course Scheduling}
\def\assignmenttitle{XCS221 Assignment \assignmentnum --- \assignmentname}
\input{macros}
\begin{document}
\pagestyle{myheadings} \markboth{}{\assignmenttitle}

% <SCPD_SUBMISSION_TAG>_entire_submission

This handout includes space for every question that requires a written response.
Please feel free to use it to handwrite your solutions (legibly, please).  If
you choose to typeset your solutions, the |README.md| for this assignment includes
instructions to regenerate this handout with your typeset \LaTeX{} solutions.
\ruleskip

\LARGE
0.a
\normalsize

% <SCPD_SUBMISSION_TAG>_0a
\begin{answer}
  % ### START CODE HERE ###
Variables: One variable $B_i$ for each button $i  \in  [1 \ldots m] $.  The domain for each variable is $[0, 1]$ or \textbf{off} and \textbf{on}.

Constraints: One constraint $f_i$ for every light bulb $j  \in [1 \ldots n]$ .

$f_i =  sum_{\text{ $j$ such that switch $B_j$ controls light $i$}}{B_j} = \text{1 mod 2}$
  % ### END CODE HERE ###
\end{answer}
% <SCPD_SUBMISSION_TAG>_0a
\clearpage

\LARGE


\normalsize

% <SCPD_SUBMISSION_TAG>_0b
\begin{answer}
  % ### START CODE HERE ###

i) There are 2 consistent assignments.  [1, 0, 1]  and [0, 1, 0]

ii)
There will be 9 calls to Backtrack
\begin{enumerate}
 \item call \# 1 assignment [{}]  
 \item call \# 2 assignment [{'X1': 0}]  
 \item call \# 3 assignment [{'X1': 0, 'X3': 0}]  
  \\ don't call for  var:X2 val:0  deltaWeight 0.0
 \item call \# 4 assignment [{'X1': 0, 'X3': 0, 'X2': 1}]  
 \item call \# 5 assignment [{'X1': 0, 'X3': 1}]  
  \\ don't call for  var:X2 val:0  deltaWeight 0.0
  \\ don't call for  var:X2 val:1  deltaWeight 0.0
 \item call \# 6 assignment [{'X1': 1}]  
 \item call \# 7 assignment [{'X1': 1, 'X3': 0}]  
  \\ don't call for  var:X2 val:0  deltaWeight 0.0
  \\ don't call for  var:X2 val:1  deltaWeight 0.0
 \item call \# 8 assignment [{'X1': 1, 'X3': 1}]  
 \item call \# 9 assignment [{'X1': 1, 'X3': 1, 'X2': 0}]  
  \\ don't call for  var:X2 val:1  deltaWeight 0.0
\end{enumerate}

  % ### END CODE HERE ###
\end{answer}
% <SCPD_SUBMISSION_TAG>_0b
\clearpage

\LARGE
2.a
\normalsize

% <SCPD_SUBMISSION_TAG>_2a
\begin{answer}
  % ### START CODE HERE ###
  % ### END CODE HERE ###
\end{answer}
% <SCPD_SUBMISSION_TAG>_2a
\clearpage

\LARGE
3.c
\normalsize

% <SCPD_SUBMISSION_TAG>_3c
\begin{answer}
  % ### START CODE HERE ###
  % ### END CODE HERE ###
\end{answer}
% <SCPD_SUBMISSION_TAG>_3c
\clearpage

% <SCPD_SUBMISSION_TAG>_entire_submission

\end{document}
